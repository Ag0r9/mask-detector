\documentclass[12pt]{article}

\usepackage[T1]{fontenc}
\usepackage[polish]{babel}
\usepackage[utf8]{inputenc}
%\usepackage{lmodern} zazwyczaj używam tej czcionki
\usepackage{mathptmx} %times new roman
\usepackage{amsmath}	%pakiet z symbolami matematycznymi
\usepackage{graphicx}	%do wstawiania rysunków
\usepackage{float}	%do umiejscowienia rysunków w dobrych miejscach
\usepackage{wrapfig} %do wstawiania obrazków obok tekstu
%\usepackage{indentfirst}	%do zaczynania pierwszego akapitu wcięciem
\usepackage{multirow}	%do tabel
\usepackage{hyperref}	%refy działają jak hyperlinki
\usepackage{caption}	%przy refie idze na góre obrazka
\usepackage[table,xcdraw]{xcolor}
\selectlanguage{polish}	%język
\graphicspath{ {./obrazki/} }	%folder gdzie znajdują się rysunki
\hypersetup{
	colorlinks   = true, %Colours links instead of ugly boxes
	urlcolor     = blue, %Colour for external hyperlinks
	linkcolor    = blue, %Colour of internal links
	citecolor    = red %Colour of citations
}
\bibliographystyle{abbrv}

\title{Komunikacja człowiek komputer - sprawozdania z aplikacji Mask Detector}
\author{Gorgoń Adam - 145278\\Grochowska Paulina - 145284}

\begin{document}
	\maketitle
	
	\section{Wstęp}
	\section{Prezentacja aplikacji}
	\section{Dane}
	\section{Użyte modele}
	\section{Działanie programu}
		\subsection{Przygotowanie zdjęcia do modelu}
		Wczytane zdjęcie jest zmieniane na tablicę jako RBG. Następnie jest układ jest zmieniany na BGR. Wynika to z faktu, że podczas dane do trenowania były wczytywane za pomocą biblioteki \href{https://opencv.org/}{OpenCV}, a w aplikacji przez \href{https://pillow.readthedocs.io/en/stable/}{Pillow}.
		
		\subsection{Detekcja twarzy}
		Ile kurwa za szlugi?\cite{marx_kapital_1962}
		\subsection{Klasyfikacja twarzy}
	\section{Podsumowanie}
	\bibliography{md_bibl.bib}
\end{document}